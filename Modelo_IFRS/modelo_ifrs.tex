% Este arquivo é uma adaptação do modelo LaTeX disponibilizado pelo UTUG (http://www.inf.ufrgs.br/utug/)
% Autor: Prof. Dr. Adriel M. Ziesemer Jr - IFRS Canoas
% O original encontra-se disponível em: https://sites.google.com/site/profadrielziesemer/home/modelolatexdetccparaoifrs
%
% Dica: Utilize o www.sharelatex.com para editar este documento. Utilize a opcão: Upload Zipped Project

\documentclass[openright]{ifrs} % utilize openright para iniciar capítulos no anverso
\usepackage[T1]{fontenc}        % pacote para conj. de caracteres correto
\usepackage[utf8]{inputenc}     % pacote para acentuaçao
\usepackage{graphicx}           % pacote para importar figuras
\usepackage{times}              % pacote para usar fonte Adobe Times
\usepackage{listings}
\usepackage{multirow}           % pacote para agrupar células em tabelas
\usepackage{scalefnt}           % pacote para redimensionar fontes em tabelas
\usepackage{amsmath}
\usepackage{rotating}           % pacote para rotacionar figuras
\usepackage{url}                % pacote para aceitar URLs (no .bib)
\usepackage{dirtytalk}          % pacote para aspas com comando \say{}
\bibliographystyle{abnt}

% ensine o latex a separar em sílabas as palavras que eventualmente ele não souber
\hyphenation{en-si-na-men-tos a-gra-de-ci-men-to de-se-nha-dos}

\author{Aluno}{Nome do}
%\author{Aluno2}{Nome do}

% edite o definicoes.sty se precisar alterar o nome do curso

\title{Modelo de Trabalho de Conclusão de Curso}

\advisor[Prof.~Dr.]{Orientador}{Nome do}
\coadvisor[Prof.~MSc.]{Co-orientador}{Nome do}

\location{Canoas}{RS}
%\date{julho}{2012} % se nao especificada, é utilizada a data atual

% palavras-chave (começar com letra maiúscula)
\keyword{Sistema}
\keyword{ABNT}
\keyword{IFRS}

% nominata
\newcommand{\nominata}{
        \MakeUppercase{\instituicao}\\
        Reitora: Prof\textsuperscript{a}.~AAAAAA\\
        Diretor do Instituto: Prof.~BBBBBBB\\
        Coordenador do curso: Prof.~CCCCCCC\\
        Bibliotecária-chefe: DDDDDDD
}

% inicio do documento
\begin{document}

% folha de rosto
\maketitle

% dedicatoria (opcional)
\clearpage
\begin{flushright}
\mbox{}\vfill
{\sffamily\itshape
Dedico este trabalho à minha família.}
\end{flushright}

% agradecimentos (opcional)
\chapter*{Agradecimentos}
Cita-los em ordem decrescente de importância.

% resumo
\begin{abstract}
Consiste na apresentação clara e concisa dos pontos relevantes do trabalho, de maneira a permitir ao leitor saber da conveniência ou não da sua leitura na íntegra. Em geral possui 3 seções: a primeira apresenta o que o trabalho trata/tema (Este trabalho apresenta...) e a importância do assunto; a segunda apresenta as novidades/metodologia utilizada no trabalho (Um novo método utilizando [...] foi desenvolvido...); a última descreve o que foi feito para validar o conteúdo (Os resultados obtidos mostraram que...) e principais conclusões.  Cada resumo ocupará no máximo uma folha e terá entre 150 e 500 palavras. Deve ser composto por uma sequência de frases completas em um mesmo parágrafo e não por uma enumeração de tópicos. Dar preferência ao uso da terceira pessoa do singular e do verbo na voz ativa (não utilizar ``eu fiz'' ou ``nós fizemos''; ao invés disso, utilizar ``foi feito''). Após, devem constar palavras-chave relativas aos assuntos do trabalho, separadas entre si por ponto.  
 \end{abstract}

% resumo na outra lingua (opcional)
\begin{englishabstract}
{Title of the Work in English}
{System. ABNT. IFRS} % Palavras Chaves: iniciar com letras maiúsculas e separar por '.'
This work has the purpose of [...]. The text in the abstract should not contain more than 500 words. Use the third person in the plural (We developed...).
\end{englishabstract}

% lista de abreviaturas e siglas
\begin{listofabbrv}{SPMD}
        \item[ILP] Programação Linear com Inteiros (\textit{Integer Linear Programming})
        \item[TADS] Curso Superior de Tecnologia em Análise e Desenvolvimento de Sistemas
\end{listofabbrv}

% lista de figuras
\listoffigures

% lista de tabelas
\listoftables

% lista de símbolos (opcional)
%\begin{listofsymbols}{$\alpha\beta\pi\omega$}
%       \item[$\sum{\frac{a}{b}}$] Somatório do produtório
%       \item[$\alpha\beta\pi\omega$] Fator de inconstância do resultado
%\end{listofsymbols}

% sumário
\tableofcontents

\chapter{Introdução}
Faça a introdução de forma a contextualizar a área na qual se enquadra o trabalho proposto. Fale sobre o problema existente na qual se tentará encontrar uma solução. Destaque a importância e relevância do trabalho.

Tome muito cuidado ao definir o título do trabalho. Ele deve refletir o principal assunto tratado. É melhor utilizar um título modesto (mais específico), mas sobre algo que foi devidamente trabalhado no texto, do que abraçar o mundo e deixar o trabalho incompleto. As vezes pode ser necessário redefini-lo quando o trabalho estiver terminado.

Como regra geral para todo o texto, evite frases muito longas (com mais de 3 linhas de extensão), isto dificulta a compreensão. Interrompa a frase com um ponto final assim que der e continue na mesma linha.

Evite utilizar ``a nível de'' ou ``em nível de''. Ao invés disto, utilize: ``com/em relação a'', ``no que concerne'', ``quanto a'', dentre outras. Também dê uma boa revisada nas regras da crase antes de começar a escrever.

Não faça afirmações fortes (ex.: nunca, sempre, impossível, ótimo,...) sem junto citar a fonte ou realizar a devida prova formal. Citar a fonte serve justamente para tirar esta responsabilidade de quem escreve um texto técnico. Também não reproduza textos de terceiros sem citar a fonte (plágio).
\section{Motivação}
Discutir as abordagens existentes e o porquê delas não serem satisfatórias.
\section{Objetivos}
Deve apresentar a nova abordagem e no que ela é superior às existentes
Alguns autores subdividem os objetivos em gerais e específicos.
\subsection{Objetivos Gerais}
Principal contribuição.
\subsection{Objetivos Específicos}
Devem ser apresentados em forma de lista, iniciando cada um com o verbo no infinitivo: esclarecer tal coisa; definir tal assunto; procurar aquilo; permitir aquilo outro, demonstrar alguma coisa etc.
\section{Organização do Texto}
O Capítulo \ref{cap:revbib} apresenta uma revisão bibliográfica sobre o tema e um estudo sobre os principais trabalhos existentes...

\chapter{Revisão Bibliográfica} \label{cap:revbib}
Apresentar o que já foi desenvolvido por outros pesquisadores (trabalhos relacionados). Priorize citações nesta ordem: periódicos, livros, teses, anais e por último qualquer outra fonte. Internacional é melhor que nacional; qualis A é melhor que B; recente é melhor do que antigo.
\section{Estudo dos Trabalhos Existentes}
Sintetizar os trabalhos citados, não precisa colocar o texto ``ao pé da letra''. Agrupe trabalhos similares e faça um resumo com suas palavras.
\section{Estado da Arte}
Apresentar o que há de mais novo (técnicas, métodos, algoritmos, ferramentas,...) sobre o assunto (publicado preferencialmente nos últimos 5 anos). Descreva os detalhes, coloque: fluxogramas, algoritmos, figuras, pontos fortes/fracos...
\section{Conclusões}
Reunir as ideias principais abordadas no capítulo.

\chapter{Pesquisa de Mercado}
Opcional, dependendo do tipo de trabalho. Serve para embasar a necessidade de algo ou coletar informações sobre o assunto. 

\chapter{Análise de Viabilidade} 
Opcional, dependendo do tipo de trabalho. Serve para provar que a abordagem proposta é viável de ser implementada.
\section{Viabilidade Técnica}
\section{Viabilidade Econômica}

\chapter{Desenvolvimento do Trabalho Proposto}
A minha contribuição.
\section{Introdução}
\section{Planejamento}
\subsection{Cronograma}
As atividades serão desenvolvidas conforme o cronograma mostrado na Tabela \ref{cronograma}.

\begin{table}
\caption{Cronograma de atividades}
\begin{center}
\begin{tabular}{l|c|c|c|c|c|c|c|c|c}
\hline \multirow{2}{*}{Tarefa}& 2012& \multicolumn{8}{c}{2013}\\
          	             & Dez & Jan & Fev & Mar & Abr & Mai & Jun & Jul & Ago \\
\hline
\hline Estudo DFM        &  X  &  X  &  X  &     &     &     &     &     &     \\
\hline Correções DRC     &     &  X  &  X  &  X  &  X  &     &     &     &     \\
\hline Finalização Astran&     &     &     &     &  X  &  X  &  X  &     &     \\
\hline Resultados        &     &     &     &     &     &     &  X  &  X  &     \\
\hline Artigos           &     &     &     &  X  &     &     &     &  X  &     \\
\hline Texto da Tese     &     &     &     &     &     &     &  X  &  X  &  X  \\
\hline Defesa	         &     &     &     &     &     &     &     &     &  X  \\
\hline 
\end{tabular}
\end{center}
\label{cronograma}
\end{table}

\subsection{Alocação dos Recursos}
\section{Execução}
\subsection{Coleta de Dados}
\subsubsection{Entrevista}
\subsubsection{Questionário}
\subsubsection{Requisitos}
\subsection{Documentação}
\subsubsection{Estrutura Geral}
\subsubsection{Casos de Uso}
O caso de uso Cadastrar Ônibus é mostrado da Tabela \ref{tabelaCadastrarOnibus}.

\begin{table}
\caption{Caso de uso - Cadastrar ônibus.}
\begin{tabular}{p{7cm}|p{7cm}}
\hline
\multicolumn{2}{c}{\bf Ação - Cadastrar ônibus} \\
\hline
\multicolumn{2}{l}{Atores:  Administrador} \\
\hline
\multicolumn{2}{l}{Pré-condição:  Autenticação do ator Administrador} \\
\hline
\multicolumn{2}{l}{Pós-condição:  Ônibus cadastrado na base de dados} \\
\hline
\multicolumn{2}{c}{\bf Fluxo de Eventos} \\
\hline
Ator & Sistema \\
\hline
1. Acessa o menu de ônibus & \\
\hline
 &  2. Exibe uma lista de ônibus cadastrados juntamente com uma opção
"adicionar ônibus" \\
\hline
3. Clica na opção "adicionar ônibus" &\\
\hline
& 4. Exibe um campo contendo id, um campo para informar número, um campo para 
seleção da acessibilidade a cadeirantes, um campo para informar a disponibilidade do 
ônibus e outro para que seja informado à qual empresa pertence.\\
\hline
5. Informa o número, a empresa à qual pertence e seleciona se é 
acessível para cadeirantes. & \\
\hline
& 6. Exibe uma mensagem de sucesso. \\
\hline
\end{tabular}
\label{tabelaCadastrarOnibus}
\end{table}

\subsubsection{Diagrama de Classes}
\subsubsection{Diagrama de Entidades e Relacionamentos}
\subsubsection{Dicionário de Dados}
\subsubsection{Especificação Detalhada dos Processos}
\subsubsection{Testes e Depurações}
\subsubsection{Considerações/Características Técnicas do Projeto}
\subsubsection{Projeto de Interface}
\subsection{Desenvolvimento}
\subsubsection{Justificativa da Escolha dos Métodos/Ferramentas}
\subsubsection{Como Foram Implementados/Adaptados os Métodos para a Solução do Problema}
\subsubsection{Divisão de Tarefas}
\subsubsection{Cronograma Executado de Atividades}
\subsubsection{Resumo das Atas de Reunião (com data)}
\subsubsection{Recursos Utilizados para Projeto/Desenvolvimento/Implantação/Treinamento/Manutenção:
Tempo, Financeiros, Materiais e Pessoais}

\chapter{Resultados}
Há trabalhos que apresentam várias metodologias e, para cada uma delas, colocam a revisão bibliográfica, desenvolvimento e resultados dentro do mesmo capítulo. Não há uma regra fixa para o corpo do trabalho, desde que tenham estes dados. 
\section{Conjuntos de Testes Utilizados}
\section{Comparações com Métodos/Ferramentas Existentes}
\section{Capturas de Tela}

\chapter{Conclusão}
Considerações finais sobre o assunto, se os objetivos foram alcançados, o que se descobriu, quais outras questões surgiram a partir dos resultados e se as hipóteses se confirmaram ou não.
\section{Contribuições}
\section{Trabalhos Futuros}
\section{Considerações Finais}

% carrega o arquivo com as bibliografias e põe o capítulo com as referências neste lugar
\bibliography{bibliografia}

% a partir daqui, todo capítulo novo é apêndice
\appendix

\chapter{Anexos e Apêndices}
Destinam-se à inclusão de informações complementares ao trabalho, mas que não são essenciais à sua compreensão. Os Apêndices devem apresentar material desenvolvido pelo próprio autor, formatado de acordo com as normas. Já os Anexos destinam-se à inclusão de material como cópias de artigos, manuais, etc., que não necessariamente precisam estar em conformidade com o modelo, e que não foram desenvolvidos pelo autor do trabalho.

%importa dicasLatexABNT.tex para o Apêndice 
\input{dicasLatexABNT.tex}

\end{document}
