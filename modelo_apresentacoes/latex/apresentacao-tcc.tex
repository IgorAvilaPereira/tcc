\documentclass[table, usenames, svgnames, dvipsnames]{beamer}
\usepackage[portuguese]{babel}
\usepackage[latin1]{inputenc}
\usepackage[absolute,overlay]{textpos}
\usepackage{array}
\usepackage{xcolor}
\usetheme{Luebeck}
\usecolortheme[named=SeaGreen]{structure}
% se��es com caixa verde claro
\usecolortheme{spruce}

\definecolor{corsubsection}{RGB}{42, 127, 79}
%\definecolor{lightpink}{RGB}{255,182,193}

\setbeamerfont{title}{size=\large}
\setbeamerfont{frametitle}{size=\large}
% com sublinhado nos titulos
%\setbeamertemplate{frametitle}{\color{black}\bfseries \vspace{10px} \insertframetitle\par\vskip-6pt\hrulefill}

\setbeamercolor{section in head/foot}{fg=white, bg=SeaGreen}
\setbeamercolor{subsection in head/foot}{fg=white, bg=corsubsection}

%For example blocks
%\setbeamercolor{block title example}{fg=red,bg=orange}
%\setbeamercolor{block body example}{fg=cyan,bg=yellow}

%For alert blocks
%\setbeamercolor{block title alerted}{fg=olive,bg=pink}
%\setbeamercolor{block body alerted}{fg=blue,bg=magenta}

%For blocks
%\setbeamercolor{block title}{fg=white,bg=blue}
%\setbeamercolor{block body}{fg=white,bg=green!40!black}

%For alert blocks
\setbeamercolor{block title alerted}{fg=white,bg=red}
\setbeamercolor{block body alerted}{fg=black,bg=pink}

\beamertemplatenavigationsymbolsempty

% Podem ser utilizadas imagens no background
\usebackgroundtemplate{\includegraphics[height=\paperheight]{figuras/rodape.png}}

% Os simbolos de navegacao nao sao necessarios
\setbeamertemplate{navigation symbols}{}
\defbeamertemplate{footline}{centered page number}
{%
	\hspace*{\fill}%
	\usebeamercolor[fg]{page number in head/foot}%
	\usebeamerfont{page number in head/foot}%
	\textcolor{white}{\insertpagenumber\,/\,\insertpresentationendpage}%
	\hspace*{\fill}\vskip2pt%
}
\setbeamertemplate{footline}[centered page number]

% Indice para cada se��o (aparece antes de cada section)
\AtBeginSection[] 
{
	\begin{frame}<handout:0>
		\frametitle{\textbf{Agenda}}
		\footnotesize{ \tableofcontents[currentsection,hideothersubsections] }
	\end{frame}
}

\DeclareGraphicsExtensions{.pdf,.jpg,.png} % compilamos apenas com pdflatex


\graphicspath{{./figuras/}} 

\title{\textbf{T�tulo}}

\subtitle{}

\author[Fulano de Tal]{\scriptsize
	Aluno: Fulano de Tal \\
	Orientador: Prof. Beltrano
}
\institute{\\[1.0mm] 
	Instituto Federal de Educa��o, Ci�ncia e Tecnologia do Rio Grande do Sul (IFRS) \\
	C�mpus Rio Grande \\
	Divis�o de Computa��o}
\date{}

\begin{document}	
	  
{%\usebackgroundtemplate{}} 
	\begin{frame}[plain]
		%\begin{figure}
		\centering
		\includegraphics[width=0.8\linewidth]{figuras/principal_superior_dc}
		%		\caption{}
		%	\label{fig:principal_superior_dc}
		%	\end{figure}	
		%		\vspace{6px}		
		\titlepage	
%		\addtocounter{framenumber}{-1}
	\end{frame}
}

\begin{frame}
\frametitle{\textbf{Agenda}}
	\footnotesize{ \tableofcontents }
\end{frame}


\section{Se��o}

\begin{frame}
	\frametitle{\textbf{Se��o}}
	\begin{block}{bloco}
		texto bloco
	\end{block}
\end{frame}


\subsection{Subse��o}

\begin{frame}
	\frametitle{\textbf{Subse��o}}
	Texto
	\begin{itemize}
		\item Item1
	\end{itemize}
\end{frame}


\subsection{Subse��o}

\begin{frame}
	\frametitle{\textbf{Subse��o}}
	\begin{itemize}
		\item Item1
	\end{itemize}
\end{frame}



\subsection{Subse��o}

\begin{frame}
	\frametitle{\textbf{Subse��o}}
\begin{enumerate}
	\item Item 1
	\item Item 2
\end{enumerate}
\end{frame}


\section{Se��o}

\begin{frame}
	\frametitle{\textbf{Se��o}}
	\begin{block}{bloco}
		texto bloco
	\end{block}
\end{frame}


\subsection{Subse��o}

\begin{frame}
	\frametitle{\textbf{Subse��o}}
	\begin{alertblock}{bloco}
		Texto
	\end{alertblock}
\end{frame}


{%\usebackgroundtemplate{}} 
	\begin{frame}[plain]
	%\begin{figure}
		\centering
		\includegraphics[width=0.8\linewidth]{figuras/principal_superior_dc}
%		\caption{}
	%	\label{fig:principal_superior_dc}
%	\end{figure}	
%		\vspace{6px}		
		\titlepage	
%		\addtocounter{framenumber}{-1}
	\end{frame}
}


\end{document}